\documentclass{article}
\usepackage{graphicx}
\usepackage{hyperref}
\usepackage[czech]{babel}
\usepackage{csquotes}
\usepackage{tabularx}

\title{IMP - Řízení spotřeby energie mikrokontroléru}
\author{Martin Slezák (xsleza26)}
\date{\today}

\begin{document}

\maketitle

\section{Úvod}

Tento projekt řeší řízení spotřeby energie mikrokontroléru, konkrétně na
mikrokontroléru FITkit3. Tento mikrokontrolér podporuje několik způsobů, jak
řídit spotřebu. Tyto způsoby budou popsány v následující sekci. Pomocí
některých z těchto prostředků byla sestavena vestavěná aplikace.

\section{Řízení spotřeby}

V této sekci budou popsány jednotlivé režimy napájení a funkcionalita
jednotlivých modulů v těchto režimech.

\subsection{Režimy napájení}

Řadič správy napájení nabízí různé možnosti, pro optimalizaci spotřeby energie
na základě požadované úrovně funkčnosti.

Podle nároků na režim zastavení aplikace existují různé režimy, které umožňují
zachování stavu, částečné vypnutí nebo úplné vypnutí vybraných logických
obvodů, případně i paměti. Stavy I/O (vstupy a výstupy) jsou zachovány ve všech
režimech.

Existují tři hlavní režimy činnosti: běhový (run), čekací (wait), zastavovací
(stop). Pomocí instrukce WFI (Wait For Interrupt) se čip přepne do čekacího
nebo zastavovacího režimu.

Pro každý běhový (run) režim existují odpovídající čekací (wait) a zastavovací
(stop) režimy. Čekací (wait) režim se podobá režimu spánku procesorů ARM.
Zastavovací (stop) jsou podobné hlubokému spánku ARM procesorů.

Existuje také režim velmi nízkého výkonu (VLPR), který umožňuje výrazně snížit
spotřebu během provozu v případě, že aplikace nepotřebuje maximální frekvenci
sběrnice.

\subsubsection{Normální režim}

\begin{tabularx}{\textwidth}{|>{\centering\arraybackslash}p{0.09\textwidth}|X|}
    \hline
    \textbf{Režim} & \textbf{Popis} \\
    \hline
    Run & Maximální výkon čipu. Defaultní režim po resetu. Zapnutý regulátor
    napětí \\
    \hline
    Wait & Jádro je uspané, periferie fungují. NVIC zůstává citlivý na
    přerušení \\
    \hline
    Stop & Čip v statickém stavu. Režim nejnižší spotřeby zachovávající stavy
    všech registrů. NVIC je zakázán, AWIC k probuzení z přerušení. Periferie
    nefungují \\
    \hline
\end{tabularx}

\subsubsection{Režim velmi nízkého výkonu}

\begin{tabularx}{\textwidth}{|>{\centering\arraybackslash}p{0.09\textwidth}|X|}
    \hline
    \textbf{Režim} & \textbf{Popis} \\
    \hline
    Run & Regulátor napětí v režimu nízké spotřeby (dostatek energie pro
    provoz při snížené frekvenci). Snížená frekvence režimu přístupu Flash.
    LVD vypnuto. Interní oscilátor poskytuje nízkovýkonový zdroj pro jádro,
    sběrnici a taky periferií \\
    \hline
    Wait & Stejné jako run s jádrem v režimu spánku \\
    \hline
    Stop & Čip v statickém stavu, vypnutý provoz LVD. Režim nejnižší spotřeby
    ADC a funkčním přerušením pinů. Periferie nefungují, lze využít RTC,
    CMP,... NVIC je zakázán, AWIC k probuzení z přerušní. \\
    \hline
\end{tabularx}

\end{document}
